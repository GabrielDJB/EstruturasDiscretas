\documentclass{article}

\usepackage[portuguese]{babel}
\usepackage[utf8]{inputenc}
\usepackage{amsmath}
\usepackage{graphicx}
\usepackage{fancyvrb}
\usepackage{fullpage}
\usepackage{float}
\usepackage{amsmath}
\usepackage{color}
\usepackage{spverbatim}


\definecolor{ogreen}{RGB}{60,128,49}

\restylefloat{table}

\title{%
  Estruturas Discretas - Segundo Trabalho\\
  \large Prof. Marcus Vinicius S. Poggi de Aragão\\
  Período de 2017.1}

\author{Gabriel Barbosa Diniz\\1511211 \and Lucas Rodrigues\\1510848 \and Mateus Ribeiro de Castro\\1213068}

\date{\today}

\begin{document}
\maketitle

\textbf{Observação$_1$}: Os códigos fontes dos algoritmos referentes aos teoremas provados seguirá em anexo em um arquivo Jupyter Notebook para melhor entendimento, compilação, execução, testes, etc.\\

\textbf{Observação$_2$}: Os arquivos de entrada e saída (walk.in e walk.out) pedidos também estarão sendo enviados juntos, cumprindo as regras e exigências pedidas no trabalho.

\section{Primeira Questão (Teorema 1)}

\textbf{Teorema 1} ($i$,$j$,$q$): Sabe-se determinar o prêmio máximo que o rei consegue coletar saindo da posição $(i,j)$ e consumindo $q$ unidades.\\

Vamos considerar um desarrolamento da matriz em 64 vértices distintos, com $v_1$ correspondente a $(1,1)$, $v_2$ a $(1,2)$, assim por diante. Os conceitos de vizinhança continuam valendo: $v_1$ tem como vizinhos $\{v_2, v_9, v_10\}$.\\

Considere, também, uma tabela cujas linhas correspondem aos vértices $v$, e as colunas ao custo $q$ restante a ser utilizado. As células da tabela serão preenchidas com o prêmio máximo $P_{max}(v_{ij},q)$, que se consegue a partir de um trajeto que inicie no vértice $v_{ij}$ e que consuma $q$ unidades.\\

\textbf{Caso Base}: Por indução em $q$, temos o caso base para $q = 0$, preencheremos a primeira coluna da tabela. Neste caso, não existem unidades para consumir, logo não poderemos sair da origem $(i,j)$. Sendo assim, o prêmio máximo para ir até $(i,j)$ será zero e para qualquer outro vértice será $-\infty$ (que representa a impossibilidade).\\

\textbf{Hipótese Indutiva}: Como hipótese indutiva, temos que o teorema é válido para $0 \leq q \leq Q$, portanto queremos provar que o teorema também é válido para $Q+1$.\\

\textbf{Passo Indutivo}: Neste caso, para cada um dos vértices $v$, devemos encontrar o prêmio máximo que pode ser obtido chegando a $v$ consumindo $Q+1$ unidades. Logo, podemos observar que, para que a condição acima seja satisfeita, no instante imediatamente anterior à chegada em $v$, estaríamos em um vértice $v_n$, vizinho de $v$, com $Q+1-q_v$ unidades consumidas, sendo $q_v$ o custo associado ao vértice $v$. Visto que o prêmio $p_v$ associado ao vértice $v$ é constante, devemos escolher $v_n$ de maneira que $P_{max}(v_n,Q+1-q_v)$ seja máximo, garantindo, assim, que $P_{max}(v,Q+1) = p_v + P_{max}(v_n,Q+1-q_v)$ também seja máximo. Vale ressaltar que, caso $Q+1-q_v < 0$, teremos que $P_{max}(v_n,Q+1-q_v) = -\infty$, uma vez que é impossivel chegar a qualquer vértice consumindo um custo total menor que zero.

\pagebreak

E assim então podemos, através da prova indutiva resolvida, podemos derivar um algoritmo genérico que corresponde a prova deste teorema. Segue abaixo então o algoritmo em \textbf{pseudocódigo} e em seguida o algoritmo em \textbf{Python}:\\

\textbf{Implementação em Pseudocódigo - Algoritmo Genérico}:

{\color{ogreen}
\begin{verbatim}
função caminhoDePremioMax(v, q)
    se q é zero
        se v é origem
            return caminho({v})
        caso contrario
            return "caminho impossivel"

    se q < 0
        return "caminho impossivel"

    Vn \(\Leftarrow\) conjunto de vizinhos de v
    caminho \(\Leftarrow\) maxPremio\{ caminhoDePremioMax( vn \(\in\) Vn, q-custo(v) )
    caminho.adicionaAoFinal(v)
    return caminho
\end{verbatim}
}

\textbf{Implementação em Python - Algoritmo Específico}:

{\color{blue}
\begin{verbatim}
def teo_3(pos, costs, rewards, energy):
  # Salvaguarda
  if pos < 0 or pos >= 64 or energy < 0:
    raise ValueError("Invalid position/energy!!")

  # Dict cujas chaves sao tuplas (posicao, energia) e cujos valores
  # sao tuplas contendo o premio maximo que o rei consegue coletar
  # comecando da posicao dada, com dada energia disponivel, e parando na
  # posicao 0 com 0 energia, e o caminho para tal
  memo = {}

  # Caso base -- q=0
  memo[(0, 0)] = (0, [0])
  for v in range(1, 64):
    memo[(v, 0)] = None

  # Hipotese indutiva e passo indutivo -- preencher a tabela
  # Para cada coluna de energia
  for q in range(1, energy+1):
    # Para cada vertice nessa coluna
    for v in range(64):
      # custo_vizinho e a coluna em que vamos olhar
      custo_vizinho = q - costs[v]
      vizinhos = find_neighbors(v)
      if custo_vizinho < 0:
        memo[(v, q)] = None
      else:
        # Filtra as celulas -- somente se nao for impossivel (not None)
        # e pega a tupla (premio, caminho) delas
        tuplas = [memo[(vizinho, custo_vizinho)] for vizinho in vizinhos if not memo[(vizinho, custo_vizinho)] == None]
        if len(tuplas) > 0:
          # O melhor_vizinho e o que tem maior premio
          melhor_vizinho = max(tuplas, key=lambda x: x[0])
          # O novo_premio e o premio do melhor vizinho somado ao do vertice em questao
          novo_premio = melhor_vizinho[0] + rewards[v]
          # O novo_caminho e o caminho do melhor vizinho acrescido do vertice em questao
          novo_caminho = melhor_vizinho[1][:] + [v]
          memo[(v, q)] = (novo_premio, novo_caminho)
        else:
          memo[(v, q)] = None

  # Com a tabela em maos, vamos encontrar o caminho comecando em 0
  # que obtenha o maior premio possivel, utilizando qualquer quantidade
  # de energia menor que a fornecida
  maior = 0
  for q in range(energy, -1, -1):
    x = memo[(0, q)]
    if x != None and x[0] > maior:
      maior = x[0]
      inst = x + (q,)
  return inst


def find_neighbors(pos):
  x = pos/8
  y = pos%8

  naive = [(x-1, y-1), (x-1, y), (x-1, y+1),
           (x,   y-1),           (x,   y+1),
           (x+1, y-1), (x+1, y), (x+1, y+1)]
  filtered = [n for n in naive if n[0]>=0 and n[1]>=0 and n[0]<8 and n[1]<8]

  return map(lambda pos: pos[0]*8+pos[1], filtered)
\end{verbatim}
}

\textbf{Testes do Algoritmo}: Os testes se encontram no arquivo Jupyter Notebook juntamente com o tempo de execução. Os testes foram realizados com instâncias pré-definidas que se encontram no arquivo $walk.in$. Os resultados encontram-se abaixo em uma tabela para melhor visualização. Segue também em anexo o arquivo walk.out.

\begin{table}[H]
\centering
\begin{tabular}{l|l|l|l|l|p{4cm}}
Instância & Tempo & Prêmio Obtido & Energia Utilizada & Energia Disponível & Caminho Encontrado\\\hline
1ª & 3.70ms & 16 & 8 & 8 & 0 1 0 1 0 1 0 1 0\\
2ª & 3.52ms & 16 & 8 & 8 & 0 1 0 1 0 1 0 1 0\\
3ª & 3.97ms & 16 & 8 & 8 & 0 1 0 1 0 1 0 1 0\\
4ª & 14.22ms & 284 & 22 & 22 & 0 9 18 27 36 44 52 60 52 60 52 60 52 60 52 60 52 44 36 27 18 9 0\\
5ª & 9.28ms & 57 & 18 & 18 & 0 8 16 25 16 25 16 25 16 25 16 25 16 25 16 25 16 8 0
\end{tabular}
\end{table}

\pagebreak

\section{Segunda Questão (Teorema 2)}


\end{document}
